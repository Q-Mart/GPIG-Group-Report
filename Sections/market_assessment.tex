


\section{Private Network}

Creating and launching the satellite network is the biggest challenge this project faces - mainly due to financial matters. Sky and Space Global have estimated their capital expenditure at \$150 million to launch a global communications network consisting of 200 nano-satellites \cite{nanosatcost} however there are other options. Cubesats \cite{cubsesat} are a cheaper, alternative option to traditional satellites and provide the functionality needed for this project. A constellation of 25 Cubesats covers the the equator from -10 to +10 latitude \cite{narayanasamy2017nanosatellites} and so it is estimated that we would need around 100 nanosatellites to cover the area of the globe needed for the courier service (as we do not require covering Antartica for example). British company Open Cosmos says that they can design, build and launch Cubesats for £500k per unit, but as the design for our Cubesats would each be the same so the overall cost would be reduced. The estimated total cost for this is £35M, however if we were to reduce the area covered initially, this would obviously reduce this cost significantly.

Our private network availability would enable our courier system to be differentiated from other current solutions such as “Courier Logistics” \cite{courierlogistics_2019} and “Speedlink Time critical” \cite{speedlinktc} as it would be far more secure. We understand that there may be issues with the connection at times due to issues such as hardware failure and this is why we would provide our customers with the option to resort back to a normal (secure) network if absolutely necessary. 

\section{Routing}

Allowing the customer the option to provide their own mapping system also adds a unique functionality, it would enable the customer to provide extra information than may be publicly available through an API such as Google Maps \cite{GoogleMaps}. The customer can also provide the system with their own route - maybe this route would allow for the use of a company plane or stopovers at company accommodation. This adds yet another customisation option for the customer's courier service that doesn't seem to exist within other services such as those companies previously mentioned.

Another feature of our routing system that allows this solution to provide a mission-critical route planning service is the security behind creating the false API requests as detailed in Section \ref{mapsarch}. This means there is an extra layer of protection over where the item being transferred is, just in case somebody manages to get inside our network. 

Current routing systems with a target market of couriers such as Onfleet \cite{onfleet} often focus more on time efficiency than a route plan based on risk and time factors combined. There has been recent research into planning routes based on these \cite{krumm2017risk} however this only covers vehicle crash rates in terms of risk not other possible risks such as travelling through dangerous countries or areas with a high probability of natural disasters. Prior to deployment, it would be worth researching further in to managing risk whilst route planning with the possible use of an external service provider.

\section{Briefcase} \label{briefcasemarket}

The innovative briefcase appears to be the first of its kind. Within the U.S, diplomatic pouches are used to transport sensitive documents and items. Our briefcase design would be able to fit the description of a diplomatic pouch if needed as the requirement according to the Vienna Convention on Diplomatic Relations is: “The packages constituting the diplomatic bag must bear visible external marks of their character and may contain only diplomatic documents or articles intended for official use” \cite{viennaConvention}. 

The briefcase’s unique electronic locking mechanism also allows for the potential of increased revenue generated from sales to wider markets such as law firms, pharmaceutical companies and other businesses wanting to have that extra level of protection on proprietary information. The only design found during research that has a similar level of security is a briefcase locked through fingerprint identification which is no longer available however seemed to have retailed at around £320 \cite{fingerprintBriefcase} \cite{biometricBriefcase}. The estimated cost of materials needed to create a briefcase is just under £100, the breakdown of these costs can be seen in Appendix \ref{tab:briefcasematcosts}. 

\section{Budget}

\subsection{Outgoings}

\begin{table}[H]
    \centering
    \begin{tabular}{|p{0.7\textwidth}|p{0.2\textwidth}|}
        \hline
        \textbf{Description} & \textbf{Cost} \\
        \hline
        Launch/creation of satellites (global cover) & £35M \\
        \hline
        Job advertising (12 employees) & £600\\
        \hline
        Background checks (12 employees) & £1000 \\
        \hline
        Initial equipment for controller and developers & £10000\\
        \hline
        Training of couriers & £5000\\
        \hline
    \end{tabular}
    \caption{Initial one-time costs}
    \label{tab:initialCosts}
\end{table}

\begin{table}[H]
    \centering
    \begin{tabular}{|p{0.7\textwidth}|p{0.2\textwidth}|}
        \hline
        \textbf{Description} & \textbf{Cost} \\
        \hline
        Salaries (couriers, controllers, developers) & £23K\\
        \hline
        Rent/other building costs for where controller and developers are based & £1000\\
        \hline
        Company Advertising & £2500\\
        \hline
        Creation of Briefcases & £10000\\
        \hline
        System Maintenance (Satellites, office equipment, server etc) & £2000\\
        \hline
        Insurance & \\
        \hline
    \end{tabular}
    \caption{Ongoing costs per month}
    \label{tab:ongoingCosts}
\end{table}

\begin{table}[H]
    \centering
    \begin{tabular}{|p{0.7\textwidth}|p{0.2\textwidth}|}
        \hline
        \textbf{Description} & \textbf{Cost} \\
        \hline
        Transport (UK) & £120\\
        \hline
        Transport (Europe) & £400\\
        \hline
        Transport (Rest of World) & £1000 \\
        \hline
        Accommodation & £120\\
        \hline
        Accommodation (Outside UK) & £140\\
        \hline
        Courier Expenses & £40\\
        \hline
    \end{tabular}
    \caption{Costs per trip per day}
    \label{tab:tripCosts}
\end{table}

\subsection{Income}

\begin{table}[H]
    \centering
    \begin{tabular}{|p{0.7\textwidth}|p{0.2\textwidth}|}
        \hline
        \textbf{Description} & \textbf{Income} \\
        \hline
        Base package & £550 \\
         \hline
        UK delivery (additional day) & £400 \\
        \hline
        Europe delivery (single day) & £850 \\
        \hline
        Europe delivery (additional day) & £700\\
        \hline
        Rest of World delivery & £1350\\
        \hline
        Use of briefcase & £60\\
        \hline
    \end{tabular}
    \caption{Income details}
    \label{tab:customerPricing}
\end{table}

As can be seen from Table \ref{tab:customerPricing}, some of our costs will be recuperated from sales to customers requiring a single/just a few deliveries. There may also be customers that require regular transfers, for which we will advertise a 5\% discount if they prepay for 100 deliveries and 10\% for 500 deliveries or more for example. This will aid the courier start-up with advertising and getting some initial funds to cover the initial costs of launching and building the satellites.

The pricing for our scheme comes from researching how other companies price their deliveries. One of the issues found when looking in to this was the need for providing full details of the delivery information prior to receiving a figure. One of the prices we found was a same day direct drive within the UK by Speedlink time critical (SLTC) which was priced at £270. However this solution does not guarantee a dedicated courier/s nor does it use a secure private network as ours does and therefore the increase in cost is justified. A similar plan was put in place for Europe where SLTC quoted £921.87 for a trip from London to Zagreb for a direct delivery. Therefore we believe that £850 is a reasonable price. These prices could also be further broken down to have more "zones" than just UK, Europe, Rest of World which is something SLTC does implement. 

Another way our solution will generate funds is through the sale of the uniquely designed briefcase. As detailed in Section \ref{briefcasemarket}, the material cost to produce each briefcase is around £100. As this item is a new addition to the secure baggage market, we would put this item to market at £400 initially, again with discounted prices if a buyer wanted to bulk-purchase the briefcases. 

Considering one of our target markets is world Governments, they often use contracts where companies bid/enter proposals for the right to fulfil the contract. Once established as a secure courier service, the solution would be entered in to these bidding processes and if won, can generate millions of pounds for the system providers. 

A promising sign for the development of our system has come from the success of Topspeed couriers \cite{topspeed}. They managed to increase from a \$1.5M turnover in their first year to \$4.5M turnover in their 4th year \cite{smesVideo}.

\section{Extension to Broader Market}
As our solution allows for so much flexibility for the customer, it is easily adapted to conquer additional markets than just diplomatic mail. One of these markets could be blood and organ transportation. Within this market, time is the most important factor and the deliveries we would provide for this market would be those with very short timeframes from collection to delivery. Another challenge would be the temperature controlled transport case/box that the medical specimens would be carried in, however these can be purchased for around £300 \cite{courierCoolers} if it was decided to pursue this sector. 

Another of the sectors that our solution would integrate well with, is the market of transferring expensive goods such as jewellery. A secure mechanism for transferring such items is needed, and our briefcase would fit the requirements for this as only the recipient has access to the contents.

Beyond diplomatic mail, this system would also be appropriate for use for business clients for the signing of confidential legal documents and contract agreements. The current system would not need any adjustments to cater to this market so this market would be targeted from the start. 
