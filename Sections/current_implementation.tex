A summary of the state of the implementation, at the time of the demonstration, and an assessment of what needs to be done in order to develop the system to an initially deployable state

\section{Frontend}
\section{Backend}
\subsection{Customer database}
\begin{flushleft}
A Docker image of Redis has been set up, and all the code interfacing it, at all three levels (data processing, RedisInterface, and API code) is complete. The API itself has also been dockerised. The processing, interface, and API-level code has been manual tested and unit tested to functional coverage. The latest version of the codebase passes all tests. Additionally, the code has been analysed using two Python PEP8 linters--- pep8 and pylint. All faults found by the linters have been removed, except for some cammelcase variable names highlighted by pylint, but used to match the names of database object attribues (for example "bannedCountries"). All code has been fully documented.
\end{flushleft}
\begin{flushleft}
The only further work required to deploy this component is further configuring the database and API server to increase the overall system security. Specifically, the oustanding tasks are:
\end{flushleft}
\begin{itemize}
    \item Set up authentication for the Redis database
    \item Disable all database commands not used by the API
    \item Obscure all database commands used by the API
    \item Set up firewall access for the Redis Docker image
    \item Encrypt all API requests and responses
\end{itemize}
\begin{flushleft}
Based on the fact that these require modifications to configuration files, but no development work, we estimate the component would be ready to deploy given at most two days of extra work.
\end{flushleft}
\section{Briefcase}
% Work to be done: work out how to destroy items

\begin{itemize}
    \item Current implemented features
    \item Work to be done
\end{itemize}